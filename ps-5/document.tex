\documentclass[11pt]{article}
\usepackage{graphicx}
\usepackage{hyperref}
\usepackage[dvipsnames, svgnames, x11names, hyperref]{xcolor}
\hypersetup{
	colorlinks,
	citecolor=Violet,
	linkcolor=Red,
	urlcolor=Blue}
%\usepackage{natbib}

\setlength{\textwidth}{6.5in}
\setlength{\headheight}{0in}
\setlength{\textheight}{8.0in}
\setlength{\hoffset}{0in}
\setlength{\voffset}{0in}
\setlength{\oddsidemargin}{0in}
\setlength{\evensidemargin}{0in}

\title{Homework 5 Solution}

\author{Nana Ama Nyamekye Darpaah}

\begin{document}
	
	\maketitle
	
	
\section{Question 1}

The formula
	\begin{equation}
	f(x) = 1 + \tanh(2x) 
	\end{equation}

can be solved analytically as 
	\begin{equation}
	f'(x) = 1 - \tanh^{2}(2x) 
	\end{equation}

Plotting the analytical version and that obtained numerically from python on the range $x = [-4, 4]$, it is seen that they are the same, with minimal error



\section{Question 2}
For a function to be maximum, it's value when its first derivative is set to $0$. For the function 
\begin{equation}
	f(x) = x^{a-1}\exp^{-x} 
\end{equation}
we see that 
\begin{eqnarray}
	f'(x) &=& (a-1)x^{a-2}\exp^{-x} - x^{a-1}\exp^{-x} = 0 \\
	f'(x) &=& \frac{a-1}{x}x^{a-1}\exp^{-x} - x^{a-1}\exp^{-x} = 0\\
	      &=&  (\frac{a-1}{x} -1)x^{a-1}\exp^{-x} = 0
\end{eqnarray}
setting the coefficient to $0$ yields
\begin{eqnarray}
	\frac{a-1}{x} - 1 = 0 \\
	x = a - 1
\end{eqnarray}

% Question 2c
For a change in variables $z = \frac{x}{c + x}$, when solved analytically yields the value of  $z = \frac{1}{2}$ at $x = c$. Hence, $c$ can be any choice provided it is equal to $x$





\section{Question 3}
\end{document}